\documentclass[12pt, UTF8]{article}

% font
\usepackage{lmodern}

% packages
\usepackage[margin=1in, a4paper]{geometry}
\usepackage{indentfirst}
\usepackage{amsmath}
\usepackage{mathtools}
\usepackage{sectsty}
\usepackage{footmisc}
\usepackage{gensymb}
\usepackage{xcolor}
\usepackage{listings}
\usepackage{caption}
\usepackage{csquotes}
\usepackage[normalem]{ulem}
\usepackage[colorlinks, allcolors=blue]{hyperref}
\usepackage[scaled=0.9]{DejaVuSansMono}

% basic config
\setlength\parindent{24pt}
\renewcommand{\thefootnote}{\fnsymbol{footnote}}
\sloppy

%\ lst listings config
\definecolor{codegreen}{rgb}{0,0.6,0}
\definecolor{codegray}{rgb}{0.5,0.5,0.5}
\definecolor{codepurple}{rgb}{0.58,0,0.82}
\definecolor{backcolour}{rgb}{0.95,0.95,0.92}

\lstdefinestyle{lstlistings_stylesheet}{
    backgroundcolor=\color{backcolour},
    commentstyle=\color{codegreen},
    keywordstyle=\color{magenta},
    numberstyle=\tiny\color{codegray},
    stringstyle=\color{codepurple},
    basicstyle=\ttfamily\scriptsize,
    breakatwhitespace=false,
    breaklines=true,
    captionpos=b,
    keepspaces=true,
    numbers=left,
    numbersep=5pt,
    showspaces=false,
    showstringspaces=false,
    showtabs=false,
    tabsize=4
}
\lstset{style=lstlistings_stylesheet}

% paper info
\title{psych-1100-lec-2}
\author{James Aaron Erang}
\date{September 16, 2022}

\begin{document}
	\maketitle
	
	\section*{Self, Society, and Culture}
	
	\textbf{Self} has been debated, discussed, and fruitfully conceptualized by different thinkers in philosophy.
	
	- On the 6th century B.C. thinkers just eventually got tired on the debated between the two component of human person, which is \textit{body and soul}, eventually renamed to \textit{body and mind}.
	
	Based on Stevens (1996), described self as:
	
	1. \textbf{Separate} -- distinct and unique (always) from other selfs, and own identity. Thus one cannot be another person.
	
	2. \textbf{Self-contained and Independent} -- not depending on or influenced by others. The self in itself can exist and has distinctness that allows it to be self contained in its own thoughts, characteristics and volition. It does not requires other selfs for it to exist.
	
	3. \textbf{Consistent} -- self's traits characteristics, tendencies, and potentialities are more or less the same. \textit{Has a personality that is enduring and expected to persist for quite some time}. Which allows it to be studied, described and measured.
	
	5. \textbf{Unitary} -- center of all experience and thoughts of the person, and is ``\textit{the chief of command hosted in an individual where all the processes, thoughts and emotions converge.}''
	
	6. \textbf{Private} -- sorts out information, feelings and emotions and thought process within the self, wherein the processes are not accessible to anyone but the self. Also since the self is always changing and dynamic, it allows external influences to take part in its shaping.
	
	\section{Social Constructionist Perspective}
	
	- Argued for a merged view of ``the person'' and ``their social context'' where the boundaries of one cannot easily be separated from the boundaries of the other" (Steven, 1996).
	
	- The self is always in participation with social life and its identity and subjected to external influences.
	
	- Self shound not be as a static entity, or multi-facted.
	
	- Self is capable to adapt itself into any circumstance it finds itself in.
	
	According to Marcel Mauss, a French antrophologist, every self has two faces: \textit{Personne} and \textit{Moi}:
	
	1. \textbf{Personne} refers to a \textit{person's sense of who he is}: \emph{coposed o the social concept of what it means to be who he is}.  And what it means to live in particular situation, such as in institution, family, religion or nationality, and how to behave given an expectations and influences from others!*.
	
	2. \textbf{Moi} - \emph{is a persons basic identity}, a sense of who he is, his body and his basic identity and biological giveness.
	
	\section{Self and the Development of the Social World}
	
	\textbf{Self} is the \emph{personality of the character that makes a person distinct.}
	
	\textbf{Social world} is \emph{where the individual creates a social interaction to other people}. For the interaction of the self to another person, his \textit{society is very important in terms of its development}.
	
	\textbf{Culture} is an \emph{expansive set of material and symbolic processes} such as world, environment, context or cultural or social systems, structures, or institutions, practices, policies and norms and values \emph{that gives direction to our behavior}. It is located in the world with the patterns of ideas, practices, institutions and artifacts, \textit{and is a product of human activity, and is considered dynamic since it is constantly invented, accumulated, and made change over time}.
	
	Culture is also not a stable set of values that reside inside people, but each individual's activity as well as thoughts, feelings, and actions of those who came before the person.
	
	Self is also dynamic in that change as the various cultural concent that engage in change, since it tries to fit in whatever culture it finds itself in.
	
	- Men and women in their growth and development engage actively in the shaping of the self.
	
	\emph{Language is a slient part of culture that has a tremendous effect in crafting of the self}, which is one of the reasons culture divide accounts for the differences on how one regards one's self It also mediates unending metamorphosis of the self.
	
	- \textbf{Language} as a both a publicly shared and privately utilized symbol system is the site where the individual and the social make and remake each other.
	
	\textbf{George Herbert Mead} is a sociologist who gives emphasis on \emph{how we form ourself image and that self image is formed through interact with other people, as well as self-awareness that is a product of social experience}.
	
	\textbf{Lev Vygotsky} is a soviet psychologist who created a theoretical framework that \emph{social interaction plays a fundamental role in of cognitive development}.
	
	According to Vygotsky, \textit{a child internalizes real-life dialogue that he had with others, which they apply with their mental-pratical problem solving, as evident by the fact that children somehow becomes what they observe and easily adapt to the personality of a particular character}.
	
	Mead and Vygotsky suggests the way that \emph{human persons develop is with the use of language acquisition and interaction with others}. It is very important in the development of self to look into the language acquisition and interaction with others, this demonstrates how important are language and interaction in terms of self development.
	
	\section{Self and Family}
	
	The impact of family is still deemed as a given in understanding of self.
	
	- The \textit{kind of family and in the resources available to the particular family such as financial and economic affects the development of self}.
	
	- Human person learns way of living and their selfhood in a family, since it is the amily that initiates the person to become and serves as the basis for the progress of the person, as the person will internalize the ways and styles they view from their family, thus the person is who he is because of his family (not entirely true I suppose).
	
	- In trying to achieve the goal of becoming a fully realized human, a child enters a system of relationships, most important which is the family, since it is the most basic unit of the society.
	
	Babies internalize ways and styles that they observe from the amily, which maybe either conscious or unconscious. Thus without family biologically or psychologically a person may not even survive or ``\textit{become a human person}''
	
	\section{Gender and Self}
	
	\textbf{Gender} \textit{is a loci of the self subject to alteration and change and development, and one of the universal dimensions in which status differences are based}, since it forms part of the selfhood that one cannot just dismiss. \emph{It refers to culturally and socially constructed differences between female and male}.
	
	- \textbf{Gender identity} -- the ability to label oneself or others as a man or woman.
	
	- \textbf{Gender role} -- expected roles, behavior, attitudes, obligations that a society assign to each sex.
	
	- \textbf{Gender stereotypes} -- strongly held ideas about the characterisitcs of male or female.
	
	- \textbf{The gendered self} -- shaped within a particular context of time and space.
	
	- \textbf{The sense of self} -- makes sure that an individual fits in a particular environment.
	
	- Self needs for the goal of truly finding one's self, self-determination, and growth.
	
	- Gender has to be personally discovered and asserted and not dictated by culture and the society.

\end{document}