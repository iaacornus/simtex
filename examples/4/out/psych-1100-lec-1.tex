\documentclass[12pt, UTF8]{article}

% font
\usepackage{lmodern}

% packages
\usepackage[margin=1in, a4paper]{geometry}
\usepackage{indentfirst}
\usepackage{amsmath}
\usepackage{mathtools}
\usepackage{sectsty}
\usepackage{footmisc}
\usepackage{gensymb}
\usepackage{xcolor}
\usepackage{listings}
\usepackage{caption}
\usepackage{csquotes}
\usepackage[normalem]{ulem}
\usepackage[colorlinks, allcolors=blue]{hyperref}
\usepackage[scaled=0.9]{DejaVuSansMono}

% basic config
\setlength\parindent{24pt}
\renewcommand{\thefootnote}{\fnsymbol{footnote}}
\sloppy

%\ lst listings config
\definecolor{codegreen}{rgb}{0,0.6,0}
\definecolor{codegray}{rgb}{0.5,0.5,0.5}
\definecolor{codepurple}{rgb}{0.58,0,0.82}
\definecolor{backcolour}{rgb}{0.95,0.95,0.92}

\lstdefinestyle{lstlistings_stylesheet}{
    backgroundcolor=\color{backcolour},
    commentstyle=\color{codegreen},
    keywordstyle=\color{magenta},
    numberstyle=\tiny\color{codegray},
    stringstyle=\color{codepurple},
    basicstyle=\ttfamily\scriptsize,
    breakatwhitespace=false,
    breaklines=true,
    captionpos=b,
    keepspaces=true,
    numbers=left,
    numbersep=5pt,
    showspaces=false,
    showstringspaces=false,
    showtabs=false,
    tabsize=4
}
\lstset{style=lstlistings_stylesheet}

% paper info
\title{psych-1100-lec-1}
\author{James Aaron Erang}
\date{September 03, 2022}

\begin{document}
	\maketitle
	
	\section*{The Self from Various Philosophical Perspective}
	
	Names, usually meaningful, are attributed to birthed progenies since it is supposed to designate the identity (thus identifier) in the world. It is thought that name signify who we are, which is true to a certain degree, and that they are intimately connected to ourself, however not in our personhood, as it is simply an identifier. It does not encompass our being, and does not equate to the self, since it is just surface of who we are due to the fact that the self is not a static thing that one is simply born with.
	
	Human inquired into the fundamental nature of the self. Along with the question of primary substratum that defines the multiplicity of things in the world. Greeks, specifically greek philosophers, are the first people to ever explore the meaning of the self and the world around them, including about the meaning of life, what does it mean to live, and the center of the universe.
	
	- Socreates and plato answered self as body, soul and spirit.
	
	- Augustine and Thomas Aquinas answered self as divine thing.
	
	- Descartes and Kant answered self as rational and thinking being
	
	- hume, ryle, amd merleau-ponty answered self as experience or experiential being.
	
	\section{Self as body, sould and spirit}
	
	Prior to Socrates, the Greek thinkers, sometimes collectively called the Pre-Socratics (denoting that some of them preceded Socrates while others existed around the same time as Socrates), preoccupied themselves with the question of the ``\textit{primary substratum, arche}'', that explains the multiplicity of things in the world, this includes Thales, Pythagoras, Parmenides, Heraclitus, and Empedocles, to name a few, were concerned with explaining what the world is really made up of, why the world is so, and what explains the changes that they observed around them, since mythological accounts propounde by poet-theologians such as Homer and Hesoid are quite non-sensical, thus these men endeavored to finally locate an explanation about the nature of change, the seeming permanence despite change, and the unity of the world amidst its diversity.
	
	Unlike Pre-Socratics, Socrates was more concerned about the problem of the ``self''. He is the first philosopher who ever engaged in a systematic questioning about the self. To Socrates, the \textit{``true task''  of the philosopher is to know oneself}.
	
	\subsection*{Socrates}
	
	As claimed by Plato in his dialogues, according to Socrates, \emph{an unexamined life is not worth living}.
	
	Socrates declared that most men, in his reckoning, wre really not fully aware of who tey were and the virtues that they were supposed to attain in order to preserve their soulds for the afterlife, in which Socrates thoght that this is the ``worst'' that can happen to anyone: ``To live but die inside.''
	
	- For socrates, every man is dualistic, composed of:
	
	- body - imperfect and impermanent
	
	- soul - permanent and perfect
	
	\subsection*{Plato}
	
	Plato is Socrate's student, who followed the school of thought and supported the idea, which he expanded. In addition to what Socrates said, Plato added that there are parts or three components to the soul:
	
	1. Rational soul - is where we think, which is the reason and intellect that governs the affairs of the human person.
	
	2. Spirited soul - emotions and feelings, that should be kept at bay.
	
	3. Appetitive soul - all the desires of the human being.
	
	In his \textit{magnum opus}, Plato emphasizes that \emph{the justice in the human person can only be attained if the three parts of the soul are working harmoniously with one another}, and only in this ``ideal'' state that the human person's soul becomes just and virtuous.
	
	\section{Self as a divine thing}
	
	\subsection*{St. Augustine}
	
	Augustine's view of the human person reflects the entire spirit of the medieval world when it comes to man. \emph{Followed Plato's view but infused with Christianity}. He agreed that man is \textit{dualistic}.
	
	- He believed that \emph{one aspect of man lives in the world and constantly desires to be with the divine. While the other one is capable of reaching immortality -- soul}.
	
	The \textit{body is bound to die on earth and the soul is to anticipate living eternally in realm of spiritual bliss in communion with God}. Since the body, as Augustine reasoned, \emph{is only capable of thriving in the imperfect, physical reality -- the world}, whereas the \textit{soul can also stay after death in an eternal realm with the all-transcendent God}.
	
	Thus, according to Augustine, \emph{the goal of man is to attain a communion of living a virtuous life on Earth as if already with God}.
	
	\subsection*{St. Thomas Aquinas}
	
	The most eminent 13th century scholar and stalwart of the medieval philosphy, followed St. Augustine's, and appended into St. Augustine's view while while adopting some ideas from Aristotle, Aquinas sadid that:
	
	- Man is composed of two parts: matter and form.
	
	- Matter or \textit{hyle} in Greek, reers to the common ``stuff'' that makes up everything in the universe, which includes the man's body.
	
	- And then form or ``morphone'' in Greek, refers to the essence of a ``substance'', which is the thing that ``makes it what it is''.
	
	In this essence, what makes a human peson a human person \emph{is his essense, not the cells, or the body that is something that ``he shares even with animals''}. Thus to him, just as for Aristotle, the \emph{soul is what animates the body, it is ``what makes us humans''}.
	
	He also believed that \emph{there is a divine factor in being alive. Thus in this school of thought they see the self as divine thing}.
	
	\section{Self as rational and thinking being}
	
	\subsection*{Rene Descartes}
	
	Is a french philospher, and the father of the modern philosophy, conceived that \emph{the human person as having a body and mind}. In his famous treatise, The Meditations of First Philosophy, Descartes claims that there is so much that we should doubt, paraphrased, ``\textit{Much of what we thing and believe, because they are not inallible, may turn out to be false.}'' One should only believe that which can pass the test of doubt (``Extraordinary claims requires extraordinary evidence'' as echoed by Dr. Carl Sagan). Thus, in the end, Descartes thought that \textit{the only thing that one cannot doubt is the existence of the self}:
	
	- \emph{Cogito ergo sum} - ``\textit{I think therefore I am}''. It means that since you are able to think that you exists, and you cannot refute it or doubt it since it is apparently clear, then it must be real.
	
	- The fact that one thinks should lead one to conclude without a trace of doubt that he exists. Then self for Descartes \emph{the human person is also combination of two distinct entities: Cogito - means the mind, then extenzia or extension which is the body}.
	
	- For him, thinking makes a man, not just having a mind.
	
	Thus for Descartes, body is simply nothing else but a machine that is attached to the mind. The human person has it but it is not what makes man a man. I at all, it is having the ability to think:
	
	\begin{displayquote}
		But what then, am I? A thinking thing. It has been said. But what is a thinking thing? It is a thing that doubts, conceives, affirms, denies, wills, refuses, that imagines also, and perceives.
	\end{displayquote}
	
	\subsection*{Immanuel Kant}
	
	As opposed to Hume, Kant thinks that the things that men perceive around them are not just randomly infused into human person without an organizing principle that regulates the relationship of all these impressions.
	
	For Kant, \emph{there is necessarily a mind that organizes the impressions that men get rom the external world, which he calls -- ``The apparatus of the mind''}.
	
	Along with the different apparatus of \textit{the mind goes the self. Without the self, one cannot organize the different impressions that one gets in relation to his own existence}.
	
	- \emph{The ``self'' is actively engaged in synthesizing all knowledge and experiences}.
	
	\section{Self as an experience or experiential being}
	
	\subsection*{David Hume}
	
	Is a Scottish philosopher and an empiricist (\textit{Empiricism is the school of thought that espouses the idea that knowledge can only be possible if it is sensed and experienced, and is only attainable by experience.}) who believes that \emph{one can know only what comes from the sense and experiences}. Humes argues that the \emph{self is not an entity over and beyond the physical body}.
	
	- \emph{Self is nothing else but a bundle of experiences}, which is categorized into two, impression and ideas. \textbf{Impression}, as defined by Hume as basic object of our experience or sensation, therefore are vivid since they are product of our direct experience with the world. \textbf{Ideas} on the other hand, are copies of expression, therefore they are not as lively and vivid as our impressions.
	
	- Self then, according to Hume, ``\emph{a bundle or collection of different perceptions, which succeed each other with an inconceivable rapidity, and are in a perpetual flux and movement}''.
	
	- Senses and experiences, empirical evidences are needed in existing, thus \textit{if you don't sense or experience something, therefore it is not real}, in the exact same manner that quantum mechanics stated that for something to exists, it must be measurable and observable, as Einstein remarked ``Did the mooon existed because a rat looked at it?''.
	
	- \emph{The self is but a collection of impressions (direct experiences, direct contact) and ideas (a concept, a feeling or thought not yet confirmed by experience)}.
	
	\subsection*{Gilbert Ryle}
	
	Solves the mind-body dichotomy that has been running for a long time in the history of thought by denying blatantly the concept o an internal, non-physical self.
	
	Ryle suggests that the \emph{self is not an entity one can locate and analyze} but simply the \textit{convenient name that people use to reer to all the behaviours that people make}.
	
	- Ryle believed that what \emph{truly matters is the behavior (how we move, act, or live), not the body}.
	
	- \emph{Self is not a single entity, but a collection of behaviors}.
	
	\subsection*{Maurice Merleau-Ponty}
	
	Is a phenomenologist who asserts that the \textit{mind-body dualism that has been on for a long time is a futile endeavor and an invalid problem}. Merleau-Ponty instead says that the \emph{mind and body are so intertwined that they cannot be separated from each other. One cannot find any experience that is not an embodied experience. Since ``all experience is embodied''}. One's body is his opening toward his existene to the world, that because of this, men are in the world.
	
	Merleau-Ponty also dismisses the Cartesian Dualism (Descartes), since for him, it is nothing else but plain misunderstanding. \emph{The living body, his thoughts, emotions, and experience are all one.}
	
	- \emph{The living body is one with his thoughts, emotions, and experiences - embodied experience} (e.g. You cannot know you are hungry without feeling it). \emph{Thus self is the experience of the body and mind}.
	
	\section*{Who Am I}
	
	It is assumed that the body guarantees the personal identity. But philosophers thought differently.
	
	Many people assume implicitly that some parts of the body correlates with the personal identity than others, which is usually the brain.
	
	Christianity run a thought experiment that asks what will happen after the death, and it imagines the separation of the body as ultimately not significant, but the ongoing of the ``soul''. Thus body does not make the person, neither the technical capabilities.
	
	Possibly, memory is part of personal identity as argued. s long as something else remained, specifically character remains the same, in some fundamental way can be claimed to the same person. Since even the memory vanishes, as long as the character remains, the person can still behave, feel, and experience in compatible ways in the future.
	
	Personal identity seems to consists not in bodily survival, nor in the memory survival, but in what is called ``character'', which is attributed to English philosopher John Locke, who wrote that ``Personal identity is made up of \emph{sameness of consciousness}. That if a demon offered of ''Remembering everything but feeling and valuing differently`` or ''Feeling and valuing the same sort of things but remembering nothing", John Locke would choose the latter.
	
	So the personal identity in essense comes down to values, inclinations, and temperament. In regards to death, which is usually thought as end of the identity, which is only true if the identify is thought as bodily survival, but if thought of large degree about character such as values, loves and hates, then in a sense ``we are granted immortality.''

\end{document}